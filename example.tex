% +++
% bibtex = "biber"
% +++
\documentclass{pkuthesis}
\usepackage[style=gb7714-2015]{biblatex}
\addbibresource{example.bib}
\usepackage[hidelinks]{hyperref}

\ThesisStyle {
    keywords-separator = {semicolon},
    title-format*      = {\zihao{-2}},
}

\ThesisInfo {
    title      = {乐队必须有键盘手},
    title*     = {Bands Must Have Keyboardists},
    author     = {千早爱音},
    student-id = {114514},
    school     = {羽丘女子学园},
    specialty  = {音乐学},
    mentor     = {
        name          = {长崎素世},
        workplace     = {月之森女子学园},
        academic-rank = {贝斯手},
    },
    abstract = {
        本文旨在探讨键盘手在现代乐队中的不可或缺性。通过分析《BanG Dream! It's MyGO!!!!!》和《Girls Band Cry》两部动画作品中乐队的组建、发展与音乐表现,特别是MyGO!!!!!、Crychic和TOGENASHI TOGEARI等乐队在演奏《春日影》和《熙熙攘攘,我们的城市》时的异同,论证键盘手在丰富乐队音色、提供和声支持、拓展音乐风格以及稳定乐队结构方面的重要作用。研究表明,键盘手不仅是音乐表现力的关键,更是乐队整体性和完整性的重要保障。
    },
    keywords = {
        乐队, 键盘手, MyGO!!!!!, Crychic, Girls Band Cry, BanG Dream!, 春日影, 熙熙攘攘,我们的城市
    },
    abstract* = {
        This paper aims to explore the indispensable role of keyboardists in modern bands. By analyzing the formation, development, and musical expression of bands in the anime series ``BanG Dream! It's MyGO!!!!!'' and ``Girls Band Cry'', particularly the similarities and differences in performances of ``Haruhikage'' and ``Nigatsu no Itsuka'' by MyGO!!!!!, Crychic, and TOGENASHI TOGEARI, this study argues for the crucial role of keyboardists in enriching band timbre, providing harmonic support, expanding musical styles, and stabilizing band structure. The research indicates that keyboardists are not only key to musical expressiveness but also vital for the overall coherence and integrity of a band.
    },
    keywords* = {
        Band, Keyboardist, MyGO!!!!!, Crychic, Girls Band Cry, BanG Dream!, Haruhikage, Nigatsu no Itsuka
    },
    grade           = {优},
    mentor-comments = {
        千早爱音同学的这篇论文深入探讨了键盘手在乐队中的重要性,选题新颖,论证充分。
        通过对《BanG Dream! It's MyGO!!!!!》和《Girls Band Cry》两部作品的细致分析,
        结合具体音乐表现案例,有力地支持了“乐队必须有键盘手”这一论点。
        论文结构严谨,语言流畅,展现了作者扎实的音乐理论功底和对乐队实践的深刻理解。
        尤其对《春日影》和《熙熙攘攘,我们的城市》的对比分析,见解独到,令人印象深刻。
        本论文达到了学士学位论文的优秀水平,同意提交答辩。
    },
}

\begin{document}

\section{引言}
在现代音乐的多元发展中,乐队作为一种重要的音乐表现形式,其构成元素对最终的音乐风格和听觉体验有着决定性的影响。传统摇滚乐队通常由吉他、贝斯、鼓和主唱组成,但随着音乐的演进,越来越多的乐队开始引入其他乐器以丰富其音色和表现力。其中,键盘手的作用常常被低估,但本文将通过对两部近期热门动画作品《BanG Dream! It's MyGO!!!!!》\cite{bangdream_mygo} 和《Girls Band Cry》\cite{girls_band_cry} 中乐队实践的深入分析,论证键盘手在乐队中不可或缺的地位。

本文将重点关注MyGO!!!!!乐队与Crychic乐队在《春日影》一曲上的不同演绎,以及MyGO!!!!!乐队与TOGENASHI TOGEARI乐队在《熙熙攘攘,我们的城市》上的表现,旨在揭示键盘乐器在和声构建、旋律补充、音色层次丰富以及乐队整体稳定性方面所发挥的关键作用。

\section{键盘手在乐队中的和声与旋律作用}

\subsection{《春日影》:Crychic与MyGO!!!!!的对比}
《春日影》一曲在《BanG Dream! It's MyGO!!!!!》中具有特殊的意义,它不仅是Crychic乐队的代表作,也成为了MyGO!!!!!乐队成立初期难以逾越的“心结”。Crychic乐队在演奏《春日影》时,虽然没有明确的键盘手角色,但其音乐编排中依然存在对和声和氛围的潜在需求。当MyGO!!!!!乐队尝试复刻这首歌曲时,由于缺乏能够提供持续和声铺垫和音色填充的乐器,歌曲的整体厚度与情感表达受到了限制。

键盘乐器,如钢琴、合成器等,能够提供宽广的音域和多样的音色,为乐队的和声结构提供坚实的基础。在《春日影》这样情感细腻、层次丰富的歌曲中,键盘的加入可以极大地增强歌曲的感染力,弥补吉他、贝斯和鼓在和声持续性上的不足,使得音乐听起来更加饱满和完整。

\subsection{《熙熙攘攘,我们的城市》:MyGO!!!!!与TOGENASHI TOGEARI的演绎}
《Girls Band Cry》中的TOGENASHI TOGEARI乐队,其音乐风格和编曲展现了对键盘音色的巧妙运用,尤其是在《熙熙攘攘,我们的城市》这首歌曲中。键盘的加入使得歌曲的氛围更加多变,从清冷的开场到激昂的高潮,键盘音色在其中起到了重要的引导和烘托作用。

相比之下,MyGO!!!!!乐队在没有键盘手的情况下,虽然通过双吉他配置在一定程度上弥补了和声的空缺,但其音色选择和层次感依然受限于传统摇滚乐队的框架。在演奏《熙熙攘攘,我们的城市》这类需要丰富音色和情绪转换的歌曲时,键盘手能够提供独特的音色纹理,如弦乐、管乐、Pad音色等,这些是吉他难以模拟的,从而极大地拓宽了乐队的音乐表现边界。键盘手不仅能提供和声,还能通过旋律线条与主唱或吉他进行呼应,增加歌曲的复杂性和趣味性。

\section{键盘手对乐队结构与稳定性的影响}
除了音乐表现力,键盘手对乐队的结构和稳定性也具有重要意义。一个优秀的键盘手能够成为乐队的“粘合剂”,将不同乐器的声音有机地融合在一起。在乐队排练和创作过程中,键盘手可以作为和声的“定海神针”,帮助其他乐手更好地理解和把握歌曲的和声走向。

在《BanG Dream! It's MyGO!!!!!》中,MyGO!!!!!乐队在早期面临的挑战之一就是成员之间的磨合和音乐理念的统一。虽然这更多是人际关系的问题,但从音乐层面来看,如果乐队中有一位能够提供稳定和声支持和音色引导的键盘手,或许能在一定程度上缓解这种不确定性,为乐队提供一个更坚实的音乐框架。键盘手能够填补乐队音色上的“空白”,使得整体声音更加平衡和完整,从而增强乐队的凝聚力。

\section{结论}
综上所述,无论是从音乐表现力、和声与旋律的丰富性,还是从乐队结构和稳定性的角度来看,键盘手在现代乐队中都扮演着不可或缺的角色。通过对《BanG Dream! It's MyGO!!!!!》和《Girls Band Cry》中乐队实践的分析,我们可以清晰地看到,有键盘手参与的乐队,其音乐往往更具层次感、更饱满、更富有感染力。键盘手不仅能够提供多样的音色和和声支持,还能在乐队内部起到协调和引导的作用,帮助乐队成员更好地协作,共同创造出更具深度和广度的音乐作品。

因此,本文坚定地认为:乐队必须有键盘手。键盘手是乐队走向成熟和多元化的重要标志,是提升乐队整体音乐品质的关键力量。

\printbibliography
\acknowledgments
感谢长崎素世导师的悉心指导与支持,以及所有为我的音乐之路提供灵感的朋友们。
\end{document}